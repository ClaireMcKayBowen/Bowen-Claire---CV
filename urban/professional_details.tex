%%%%%%%%%%%%%%%%%%%%%%%%%%%%%%%%%%%%%%%%%%%%
% Research Experience
%%%%%%%%%%%%%%%%%%%%%%%%%%%%%%%%%%%%%%%%%%%%

\textbf{Urban Institute} \hfill \textit{Dec. 2019 - Present}\\
    \textit{Principal Research Associate and Statistical Methods Group Lead \hfill Oct. 2021 - Present}\\
    \textit{Lead Data Scientist, Privacy and Data Security \hfill Dec. 2019 - Sept. 2021}
    \begin{itemize}
        \item Leading a multi-year project with the IRS to create synthetic versions of the income tax return database, assess the quality and privacy guarantee of the generated synthetic data, and develop new methods for researchers to access confidential administrative data safely and securely.
        \item Co-leading a project to design and implement the Restricted Use Data Access Program that will safely promote and expand restricted-use Department of Labor data access to facilitate timely, accurate, and informative analysis, research, and program evaluation.
        % \item Managing multiple data privacy and confidentiality projects on generating differentially private synthetic data on employer and employee industry data in rural communities in collaboration with the Bureau of Labor Statistics and Bureau of Economic Analysis.
        \item Leading the Statistical Methods Group that provides consulting services for Urban Institute, such as reviewing proposals, brainstorming ideas, and identifying appropriate statistical analyses.
        % \item Developing a body of work on data privacy and confidentiality communications, which includes blogs, research briefs, and a book.
        \item Develops and hosts training on data privacy and confidentiality concepts and methods for various clients, such as Allegheny County and Bureau of Economic Analysis.
        \item Designed an interactive dashboard to assess and evaluate the 2020 Decennial Census population results in collaboration with Massive Data Institute at Georgetown University.
        \item Created a set of tools and resources on how to virtually teach kids about data, data science, and data visualization for 3rd-12th grade teachers, resulting in multiple publications.
    \end{itemize}

\workspace
    \textbf{Stonehill College - School of Business} \hfill \textit{June 2022 - Present}\\
    \textit{Adjunct Professor}
    \begin{itemize}
        \item Teaching duties included preparing a syllabus, creating a new course, lecturing, assigning homework, writing exams, and holding office hours.
    \end{itemize}

\workspace
    \textbf{Los Alamos National Laboratory (LANL) - Los Alamos, NM} \hfill \textit{June 2018 - Dec. 2019}\\
    \textit{Postdoctoral Research Associate, \underline{Advisors}: \href{http://www.lanl.gov/expertise/profiles/view/joanne-wendelberger}{Dr. Joanne Wendelberger and Dr. Earl Lawrence}}
    \begin{itemize}
        \item Evaluated various differentially private data synthesis methods and quality metric algorithms to assess practical applications.
        \item Developed new methods of functional data analysis, human-in-the-loop, and uncertainty quantification for iterative design.
        \item Investigated the reliability of computer hardware such as cosmic ray effects on supercomputers e.g. Trinity (no. 5 fastest supercomputer upon installment).
        \item Analyzed and evaluated Texas Pre-K to college longitudinal education data to identify attributes for success.
    \end{itemize}

\workspace
    \textbf{University of Notre Dame - Notre Dame, IN, Dept. of ACMS} \hfill \textit{Aug. 2013 - May 2018}\\
    \textit{Graduate Research Assistant, \underline{Advisor}: \href{http://acms.nd.edu/people/faculty/fang-liu/}{Dr. Fang Liu} \hfill Aug. 2013 - May 2018}\\
    \textit{Probability and Statistics Instructor \hfill Jan. 2015 - May 2015}
    \begin{itemize}
        \item Develop new methods of data privacy and confidentiality in big data using Bayesian Statistics through R Programming; allowing education, economics, medical research, national security, and other areas to share data without disclosing personal information about participants.
        \item Teaching duties included preparing a syllabus, lecturing, assigning homework, writing exams, and holding weekly office hours.
    \end{itemize}

\workspace
    \textbf{U.S. Census Bureau - Suitland, MD} \hfill \textit{Aug. 2016 - Oct. 2016}\\
    \textit{Graduate Research Assistant, \underline{Advisor}: Dr. Tommy Wright and Dr. Martin Klein}
    \begin{itemize}
        \item Implemented methods of data privacy and confidentiality to U.S. Census Bureau data; specifically, applying model-based differentially private data synthesis methods.
    % \item Awarded the opportunity to intern at the U.S. Census Bureau through the National Science Foundation Graduate Research Internship Program.
    \end{itemize}
    
% \vspace{4pt}
%     \cventry{}{Office of Grants and Fellowships Consultant}{University of Notre Dame - Notre Dame, IN, Graduate School}{\textnormal{\textit{July 2014 - May 2016}}}{}
%     {
%     % \begin{itemize}
%     %     \item Successfully advised several NSF GRFP winners from various STEM fields.
%     % \end{itemize}
%     }

\workspace
    \textbf{Los Alamos National Laboratory (LANL) - Los Alamos, NM} \hfill \textit{May 2015 - July 2015}\\
    \textit{Graduate Research Assistant, \underline{Advisor}:  \href{http://www.lanl.gov/expertise/profiles/view/joanne-wendelberger}{Dr. Joanne Wendelberger}}
    %  and \href{http://www.lanl.gov/expertise/profiles/view/lawrence-ticknor}{Mr. Lawrence Ticknor}
    \begin{itemize}
        \item Improved an in situ method for approximating complex computer simulations, resulting in a publication. 
    % (publication \#3).
        \item Created a R package and Shiny App of the developed method as additional learning tools for researchers.
    \end{itemize}
    
\workspace
    \textbf{Indiana University of Health - Goshen, IN Center for Cancer Care} \hfill \textit{Jan. 2014 - Nov. 2014}\\
    \textit{Graduate Research Assistant, \underline{Advisor}: \href{http://iuhealth.org/find-a-doctor/physician/69771/}{Dr. James A. Wheeler}}
    \begin{itemize}
        \item Wrote programs in R to investigate clinical data sets using survival analysis, resulting in a publication.
    % (publication \#2).
        \item Created R Shiny Online Applications as an additional learning tool for published paper.
    \end{itemize}
    
\workspace
    \textbf{Idaho State University - Pocatello, ID, Dept. of Physics} \hfill \textit{Sept. 2010 - May 2013}\\
    \textit{Physics Proctor and Grader \hfill Sept. 2012 - May 2013}\\
    \textit{Laboratory Physics Teaching Assistant \hfill Jan. 2011 - May 2013}\\
    \textit{Physics Tutor \hfill Sept. 2010 - May. 2013}
    \begin{itemize}
        \item Proctored and graded algebra-based introductory physics class exams.
        \item Taught and tutored introductory algebra and calculus-based physics laboratories.
    \end{itemize}
    
\workspace
\newpage
    \textbf{University of Illinois - Urbana-Champagne, IL, Dept. of Physics} \hfill \textit{Summers of 2011 and 2012}\\
    \textit{Undergraduate Research Assistant, \underline{Advisor}: Dr. Jose Mestre \hfill May 2012 - Aug. 2012}\\
    \textit{Research Experience for Undergraduate Researcher, \underline{Advisor}: Dr. Taekjip Ha \hfill May 2011 - Aug. 2011}
    \begin{itemize}
        \item Created a computer program to record subjects' reaction times, identifying physics-related or non-related changes in images for statistical analysis to determine expertise in workplace competency.
        \item Prepared delicate DNA, RNA, and protein samples for analysis and conducted experiments on optical benches using florescence resonance energy transfer technique to gather data for further analysis, using MATLAB to discover DNA, RNA, and protein interaction rates.
    \end{itemize}

\workspace
    \cventry{}{\href{http://www.int.washington.edu/REU/2010/2010_photos.html}{Research Experience for Undergraduate Researcher}}{University of Washington - Seattle, WA, Dept. of Physics}{\textnormal{\textit{May 2010 - Aug. 2010}}}{\underline{Advisor}: Dr. Peter Shaffer}
    {
    \begin{itemize}
    \item Conducted statistical analysis on student response data to discover student misconceptions on physics topics through \textit{Tutorials in Introductory Physics} book, improving student understanding by more than 10\%.
    \end{itemize}
    }

\workspace
    \cventry{}{Laboratory Technician}{Idaho State University - Pocatello, ID, Environmental Monitoring Lab}{\textnormal{\textit{May 2008 - May 2010}}}{}
    {
    \begin{itemize}
    \item Monitored Idaho National Laboratory's radiation emissions in the environment, performing these analyses:
    \begin{itemize}
    \item Gross alpha and beta analysis of air particulate filters and water samples on proportional counters.
    \item Gamma-emitting radionuclides of charcoal, air particulate filters, and water on gamma spectrometers.
    \item Tritium and low level tritium on liquid scintillation counter.
    \end{itemize}
    \end{itemize}
    }
    